\section{What is \textsf{Zenodo}?}

Zenodo is a general-purpose open-access repository operated by \textsc{cern}, where researchers can deposit research papers, datasets, research software, reports, and any other research related digital artifacts. For each submission, a persistent \textit{digital object identifier} (DOI) is minted, which makes the stored items easily citeable.

It has a \href{https://zenodo.org/}{webage} very well \textit{self-documented}, with \href{https://about.zenodo.org/}{detailed information} about the operation and a complete help guide with \href{https://help.zenodo.org/}{FAQ}, as well as a field to \href{https://zenodo.org/support}{contact} with them.
% It is developed under the European \href{https://www.openaire.eu/faqs#article-id-1100}{OpenAIRE} program, which  

\section{How to upload software?}

A very clean way to upload software is to develop it in \textsf{GitHub}, create a release and then publish it in Zenodo.


