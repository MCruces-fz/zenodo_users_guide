%% **************************** %%

%%  --------  HEADERS --------  %%

\documentclass[a4paper]{article}
\usepackage[utf8]{inputenc}
\usepackage[english]{babel}

%%  -------  PACKAGES  -------  %%

\usepackage{graphicx}
\usepackage{booktabs}
\usepackage{amsmath}
\usepackage{listings}
\usepackage{xcolor}
\usepackage{hyperref}
\usepackage{breakurl}

%%  ------  DEFINITIONS ------  %%

\include{definitions}

%% **************************** %%



%% **************************** %%
%%  B E G I N - D O C U M E N T %%
%% **************************** %%

\begin{document}

%% ******** T I T L E ********* %%

\begin{centering}
    \begin{minipage}[c]{0.4\linewidth}
        \centering
    \includegraphics[height=5\baselineskip]{LogoUSC}
    \end{minipage}
    
    	\vspace{1.5cm}
        {\huge \textsc{Users Guide} \\ Publish software easily in Zenodo\par}
    	\vspace{0.25cm}
    
    	
    	\noindent\rule{\textwidth}{1pt}
    	\vspace{0.5cm}
        
        Miguel \textsc{Cruces}
    	
    	\vspace{0.5cm}
    	{\large\bfseries LabCAF / IGFAE  \par Universidade de Santiago de Compostela}
    
    	\vspace{1cm}
\end{centering}


% \tableofcontents

\section{What is \textsf{Zenodo}?}

Zenodo is a general-purpose open-access repository operated by \textsc{cern}, where researchers can deposit research papers, datasets, research software, reports, and any other research related digital artifacts. For each submission, a persistent \textit{digital object identifier} (DOI) is minted, which makes the stored items easily citeable.

It has a \href{https://zenodo.org/}{webage} very well \textit{self-documented}, with \href{https://about.zenodo.org/}{detailed information} about the operation and a complete help guide with \href{https://help.zenodo.org/}{FAQ}, as well as a field to \href{https://zenodo.org/support}{contact} with them.
% It is developed under the European \href{https://www.openaire.eu/faqs#article-id-1100}{OpenAIRE} program, which  

\section{How to upload software?}

A very clean way to upload software is to develop it in \textsf{GitHub}, create a release and then publish it in Zenodo.




Comezo pola resposta máis fácil: se podes facer ambos cursos, perfecto! Podes botar unha ollada ás condicións e as datas? Poño en copia a Néstor e Héctor. Cómpre logo explicarllo a Hans e José Ángel.

 

Efectivamente ese é o texto da proposta que enviamos coa solicitude de financiamento para o que rematou sendo a prórroga do teu contrato.

 

Sigo sen ser quen de explicarme... volvemos a falar cando queiras, pero creo que tamén pode falar con Héctor con quen o comentei onte.

 

O feito de empregar o teu traballo en TRASGO e NEXT como “caso de estudo” é a aproximación máis directa e doada porque xa traballas nese software. Podería ser calquera outro. Por exemplo, o de reconstrución PET desenvolto por Josh.

 

Non interfire co que se estea a facer para TRASGO. O obxectivo é preparar uns contidos mínimos sobre o que é o software Open Source, como se contribúe a ese tipo de desenvolvemento e como o fai o IGFAE para que no sitio web do IGFAE apareza de xeito obvio que somos un centro de investigación comprometido con esa filosofía. De feito, non coñezo ningún desenvolvemento que se faga con outra idea no instituto. Iso é obvio para todos/as nos, pero non se ve.

 

Volvemos a comentar cando queiras. Creo que o máis sinxelo é quedar un día con Héctor.

 

Como comentou Néstor no seu momento, estaría ben, e se recolle tamén como obxectivo, ter un inventario de todos os desenvolvementos OS do IGFAE. Pero para iso non vexo que faga falla en principio moito máis que ter o enlace ao repositorio. E volvo ao exemplo que nos enviou Josh:


\end{document}

